% This is a borrowed LaTeX template file for lecture notes for CS267,
% Applications of Parallel Computing, UCBerkeley EECS Department.

% To familiarize yourself with this template, the body contains
% some examples of its use.  Look them over.

\documentclass[a4paper]{article}

%
% ADD PACKAGES here:
%

\usepackage{amsmath,amsfonts,graphicx,multicol}
\usepackage[margin=1in]{geometry}
\usepackage{amsthm}
\usepackage{graphicx}
\graphicspath{ {./images/} }
\usepackage{bm,nicefrac}
\usepackage{hyperref}

\setlength{\parskip}{\baselineskip} % Set space between paras
\setlength{\parindent}{0pt} % No para indentation

\newtheorem{theorem}{Theorem}[section]
\newtheorem{corollary}[theorem]{Corollary}
\newtheorem{lemma}[theorem]{Lemma}
\newtheorem{definition}[theorem]{Definition}
\newtheorem{remark}[theorem]{Remark}
\newtheorem{problem}{Problem}

\begin{document}

\pagestyle{myheadings}
   \thispagestyle{plain}
   \newpage
   \noindent
   \begin{center}
   \framebox
   {
      \vbox{\vspace{2mm}
        \hbox to 6.28in { {\Large \hfill Linear algebra \hfill} } %% Fill here
        \vspace{2mm}
        \hbox to 6.28in { {\it Chaitanya Kharyal \hfill} } %% Fill here
        \vspace{2mm}}
   }
   \end{center}
   \markboth{}{} %% Fill here

\section{Notation:}
We have tried to use uppercase letters ($\mathbf{A}$) for matrices, lowercase bold letters ($\mathbf{x}$) for vectors and lowercase letters ($x$) for scalars (constants and variables). We have tried to stick to this notation as much as possible but there might be some unintentional slips and we apologise for that.

\section{Topics to be covered}
\begin{itemize}
    \item $\mathbf{A} = \mathbf{L}\mathbf{U}$
    \item Eigen values and Eigen vectors
    \item Singular Value Decomposition (SVD)
\end{itemize}

\section{$\mathbf{A} = \mathbf{L}\mathbf{U}$}
    \subsection{representing row operations as matrices}
        suppose we take the matrix,
        \begin{center}
            $
            \mathbf{A} = 
            \begin{bmatrix}
                a & b & c & d \\ 
                e & f & g & h \\
                i & j & k & l \\
                m & n & o & p 
            \end{bmatrix}
            $
        \end{center}
        
        And, we want to perform the operation $R_2 = R_2 - R_1$. (Note that while converting a matrix to Row Echelon form, we always perform the operations of type $R_i = R_i + cR_j$, where $i > j$). What will be the matrix for this operation?
        
        \subsubsection{Row picture of matrix multiplication}
            Just like the Column picture of matrix multiplication, you can interpret a matrix multiplication as Row picture. Suppose we have a matrix $\mathbf{A}$ and we multiply a row vector $\mathbf{x'}$ to its left. Formally,
            
            \begin{center}
                $
                \mathbf{x'}\mathbf{A} = \mathbf{b'}
                $
                
                $
                \begin{bmatrix}
                x_1   &
                x_2   &
                \dots &
                x_m 
                \end{bmatrix}
                \begin{bmatrix}
                a_{1,1} & a_{1,2} & \dots & a_{1,n} \\ 
                a_{2,1} & a_{2,2} & \dots & a_{2,n} \\
                \vdots  & \vdots  &       & \vdots  \\
                a_{m,1} & a_{m,2} & \dots & a_{m,n} 
                \end{bmatrix}
                 = 
                \begin{bmatrix}
                b_1   &
                b_2   &
                \dots &
                b_n
                \end{bmatrix}
                $
                
                $
                \begin{bmatrix}
                x_1   &
                x_2   &
                \dots &
                x_m 
                \end{bmatrix}
                \begin{bmatrix}
                \rule[.5ex]{2.5em}{0.4pt} & \mathbf{r_1} & \rule[.5ex]{2.5em}{0.4pt} \\ 
                \rule[.5ex]{2.5em}{0.4pt} & \mathbf{r_2} & \rule[.5ex]{2.5em}{0.4pt} \\
                   & \vdots  &  \\
                \rule[.5ex]{2.5em}{0.4pt} & \mathbf{r_m} & \rule[.5ex]{2.5em}{0.4pt}
                \end{bmatrix}
                 = 
                \begin{bmatrix}
                b_1   &
                b_2   &
                \dots &
                b_n
                \end{bmatrix}
                $ \dots\dots \texttbf{(1)}
            \end{center}
            Where,
            \begin{center}
                $
                    \mathbf{r_i} = 
                    \begin{bmatrix}
                    a_{i,1}   &
                    a_{i,2}   &
                    \dots     &
                    a_{i,n} 
                    \end{bmatrix}
                $
            \end{center}
            
            Then, equation \textbf{(1)} can be written as,
            \begin{center}
                $
                \displaystyle\sum_{i=1}^{m}x_i\textbf{r}_i = \textbf{b} \dots\dots \textbf{(2)}
                $
            \end{center}
        
            Verify that \textbf{(2)} is correct.
            
            This can easily be extended to multlipication of two matrices (instead of a matrix and a vector). For example,
            
            \begin{center}
                $
                    \begin{bmatrix}
                x_1   & x_2   & \dots & x_m \\
                y_1   & y_2   & \dots & y_m
                \end{bmatrix}
                \begin{bmatrix}
                \rule[.5ex]{2.5em}{0.4pt} & \mathbf{r_1} & \rule[.5ex]{2.5em}{0.4pt} \\ 
                \rule[.5ex]{2.5em}{0.4pt} & \mathbf{r_2} & \rule[.5ex]{2.5em}{0.4pt} \\
                   & \vdots  &  \\
                \rule[.5ex]{2.5em}{0.4pt} & \mathbf{r_m} & \rule[.5ex]{2.5em}{0.4pt}
                \end{bmatrix}
                 = 
                \begin{bmatrix}
                \displaystyle\sum_{i=1}^{m}x_i\textbf{r}_i \\
                \displaystyle\sum_{i=1}^{m}y_i\textbf{r}_i
                \end{bmatrix}
                $
            \end{center}
            
        \subsubsection{Row operations as matrices}
            Now that we know how to represent matrix multlipication in the row picture, we can continue our discussion on finding the matrix for operation $R_2 = R_2 - R_1$. Using the row picture, we know the vector that represents $R_2 - R_1$ is,
            
            \begin{center}
                $
                \begin{bmatrix}
                    -1 & 1 & 0 & 0 
                \end{bmatrix}
                $
            \end{center}
            
            But, we also want to keep all the other rows as they are. Therefore, the matrix becomes,
            
            \begin{center}
                $
                    \begin{bmatrix}
                    1  & 0 & 0 & 0 \\
                    -1 & 1 & 0 & 0 \\
                    0  & 0 & 1 & 0 \\
                    0  & 0 & 0 & 1 
                    \end{bmatrix}
                $
            \end{center}
            
            You can multiply this matrix to $\mathbf{A}$ and see that this, in fact, performs the operation $R_2 = R_2 - R_1$. Note that this is a \textbf{Lower Triangular matrix}.
            
\section{Eigen values and Eigen vectors}

\section{SVD}

\end{document}
