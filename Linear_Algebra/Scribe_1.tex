% This is a borrowed LaTeX template file for lecture notes for CS267,
% Applications of Parallel Computing, UCBerkeley EECS Department.

% To familiarize yourself with this template, the body contains
% some examples of its use.  Look them over.

\documentclass[a4paper]{article}

%
% ADD PACKAGES here:
%

\usepackage{amsmath,amsfonts,graphicx,multicol}
\usepackage[margin=1in]{geometry}
\usepackage{amsthm}
\usepackage{graphicx}
\graphicspath{ {./images/} }
\usepackage{bm,nicefrac}
\usepackage{hyperref}

\setlength{\parskip}{\baselineskip} % Set space between paras
\setlength{\parindent}{0pt} % No para indentation

\newtheorem{theorem}{Theorem}[section]
\newtheorem{corollary}[theorem]{Corollary}
\newtheorem{lemma}[theorem]{Lemma}
\newtheorem{definition}[theorem]{Definition}
\newtheorem{remark}[theorem]{Remark}

\begin{document}

\pagestyle{myheadings}
   \thispagestyle{plain}
   \newpage
   \noindent
   \begin{center}
   \framebox
   {
      \vbox{\vspace{2mm}
        \hbox to 6.28in { {\Large \hfill Linear algebra \hfill} } %% Fill here
        \vspace{2mm}
        \hbox to 6.28in { {\it Chaitanya Kharyal\hfill} } %% Fill here
        \vspace{2mm}}
   }
   \end{center}
   \markboth{}{} %% Fill here

\section{Topics to be covered}
\begin{itemize}
    \item System of linear equations
    \item Row Echelon Form
    \item Reduced Row Echelon Form
\end{itemize}

\section{System of linear equations}
\subsection{Representing linear equations in terms of matrices and vectors}
\textit{(covered in T8)}

While solving the system of linear equations using LA, the first question that arises is \textit{"How do we represent Linear Equations using matrices?".} This relatively not so simple question has a simple solution. Consider a system of linear equations in two variables,
\begin{center}
    $ax + by = c$
    
    $dx + ey = f$
\end{center}

In order to represent this simple system of equations, we make a matrix using all the coefficients...
\begin{center}
    $A = \begin{bmatrix}
        a & b \\
        d & e 
    \end{bmatrix}$
\end{center}

We also make two vector using the variables and the constants respectively...
\begin{center}
    $\textbf{x} = \begin{bmatrix}
        x \\
        y 
    \end{bmatrix}$ \hspace{5px} 
    and \hspace{5px}
$\textbf{b} = \begin{bmatrix}
        c \\
        f 
    \end{bmatrix}$ 
\end{center}

Now, our linear equation will become:
\begin{center}
    $A\textbf{x} = \textbf{b}$
\end{center}

\subsection{Column picture of matrix multiplication}
\textit{(covered in T8)}

The fundamental problem in LA is to solve $A\textbf{x} = \textbf{b}$ for $\textbf{x}$. We will now see when is the solution for this equation possible.

We write our linear equation in matrix form,
\begin{center}
    $ 
    \begin{bmatrix}
    a_{1,1} & a_{1,2} & \dots & a_{1,n} \\ 
    a_{2,1} & a_{2,2} & \dots & a_{2,n} \\
    \vdots  & \vdots  &       & \vdots  \\
    a_{m,1} & a_{m,2} & \dots & a_{m,n} 
    \end{bmatrix}
    \begin{bmatrix}
    x_1   \\ 
    x_2   \\
    \vdots\\
    x_n 
    \end{bmatrix} = 
    \begin{bmatrix}
    b_1   \\ 
    b_2   \\
    \vdots\\
    b_n 
    \end{bmatrix}
    
    \implies \begin{bmatrix}
    |   & |       &       & |     \\ 
    c_1 & c_2     & \dots & c_n   \\
    |   & |       &       & |        
    \end{bmatrix}
    \begin{bmatrix}
    x_1   \\ 
    x_2   \\
    \vdots\\
    x_n 
    \end{bmatrix} = 
    \begin{bmatrix}
    b_1   \\ 
    b_2   \\
    \vdots\\
    b_n 
    \end{bmatrix}
    $  \dots\dots \textbf{(1)}
\end{center}

Where,
\begin{center}
    $
    c_i = 
    \begin{bmatrix}
    a_{1,i}   \\ 
    a_{2,i}   \\
    \vdots    \\
    a_{m,i} 
    \end{bmatrix}
    $
\end{center}

We can rewrite the equation \textbf{(1)} as,

\begin{center}
    $
    \displaystyle\sum_{i=1}^{n}x_i\textbf{c}_i = \textbf{b} \dots\dots \textbf{(2)}
    $
\end{center}
Verify that (2) is correct.

Here, $\displaystyle\sum_{i=1}^{n}x_i\textbf{c}_i$ is the span of column vectors of $\textbf{A}$, since $\textbf{x}$ is unknown. And for L.H.S and R.H.S of equation (2) to be equal for some $\textbf{x}$, $\textbf{b}$ should belong to the column space of $\textbf{A}$.

If $\textbf{b}$ doesn't belong to the column space of $\textbf{A}$, there will be no solution for the equation. In that case if we want to approximate the solution for this equation, we have to use the trick that was taught in the last tutorial. And the solution, then, comes out to be,
\begin{center}
    $
    \textbf{x} = (\textbf{A}^T\textbf{A})^{-1}\textbf{A}^T\textbf{b}
    $
\end{center}

But if $\textbf{b}$ belongs to $\textbf{A}$, how to solve the equation? This question still remains and before looking into how to solve the equations, we need to learn some more concepts.

\section{Row Echelon Form}
\subsection{What is RE form?}

\subsection{Reduced RE form}

\section{Solution of equations using RE form}

\end{document}
