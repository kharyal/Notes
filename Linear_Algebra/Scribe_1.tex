% This is a borrowed LaTeX template file for lecture notes for CS267,
% Applications of Parallel Computing, UCBerkeley EECS Department.

% To familiarize yourself with this template, the body contains
% some examples of its use.  Look them over.

\documentclass[a4paper]{article}

%
% ADD PACKAGES here:
%

\usepackage{amsmath,amsfonts,graphicx,multicol}
\usepackage[margin=1in]{geometry}
\usepackage{amsthm}
\usepackage{graphicx}
\graphicspath{ {./images/} }
\usepackage{bm,nicefrac}
\usepackage{hyperref}

\setlength{\parskip}{\baselineskip} % Set space between paras
\setlength{\parindent}{0pt} % No para indentation

\newtheorem{theorem}{Theorem}[section]
\newtheorem{corollary}[theorem]{Corollary}
\newtheorem{lemma}[theorem]{Lemma}
\newtheorem{definition}[theorem]{Definition}
\newtheorem{remark}[theorem]{Remark}

\begin{document}

\pagestyle{myheadings}
   \thispagestyle{plain}
   \newpage
   \noindent
   \begin{center}
   \framebox
   {
      \vbox{\vspace{2mm}
        \hbox to 6.28in { {\Large \hfill Linear algebra \hfill} } %% Fill here
        \vspace{2mm}
        \hbox to 6.28in { {\it Chaitanya Kharyal\hfill} } %% Fill here
        \vspace{2mm}}
   }
   \end{center}
   \markboth{}{} %% Fill here

\section{Topics to be covered}
\begin{itemize}
    \item System of linear equations
    \item Row Echelon Form
    \item Reduced Row Echelon Form
\end{itemize}

\section{System of linear equations}
\subsection{Representing linear equations in terms of matrices and vectors}
\textit{(covered in T8)}

While solving the system of linear equations using LA, the first question that arises is \textit{"How do we represent Linear Equations using matrices?".} This relatively not so simple question has a simple solution. Consider a system of linear equations in two variables,
\begin{center}
    $ax + by = c$
    
    $dx + ey = f$
\end{center}

In order to represent this simple system of equations, we make a matrix using all the coefficients...
\begin{center}
    $A = \begin{bmatrix}
        a & b \\
        d & e 
    \end{bmatrix}$
\end{center}

We also make two vector using the variables and the constants respectively...
\begin{center}
    $\textbf{x} = \begin{bmatrix}
        x \\
        y 
    \end{bmatrix}$ \hspace{5px} 
    and \hspace{5px}
$\textbf{b} = \begin{bmatrix}
        c \\
        f 
    \end{bmatrix}$ 
\end{center}

Now, our linear equation will become:
\begin{center}
    $A\textbf{x} = \textbf{b}$
\end{center}

\end{document}
