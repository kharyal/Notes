% This is a borrowed LaTeX template file for lecture notes for CS267,
% Applications of Parallel Computing, UCBerkeley EECS Department.

% To familiarize yourself with this template, the body contains
% some examples of its use.  Look them over.

\documentclass[a4paper]{article}

%
% ADD PACKAGES here:
%

\usepackage{amsmath,amsfonts,graphicx,multicol}
\usepackage[margin=1in]{geometry}
\usepackage{amsthm}
\usepackage{graphicx}
\graphicspath{ {./images/} }
\usepackage{bm,nicefrac}
\usepackage{hyperref}

\setlength{\parskip}{\baselineskip} % Set space between paras
\setlength{\parindent}{0pt} % No para indentation

\newtheorem{theorem}{Theorem}[section]
\newtheorem{corollary}[theorem]{Corollary}
\newtheorem{lemma}[theorem]{Lemma}
\newtheorem{definition}[theorem]{Definition}
\newtheorem{remark}[theorem]{Remark}

\begin{document}

\pagestyle{myheadings}
   \thispagestyle{plain}
   \newpage
   \noindent
   \begin{center}
   \framebox
   {
      \vbox{\vspace{2mm}
        \hbox to 6.28in { {\Large \hfill Learning Theory \hfill} } %% Fill here
        \vspace{2mm}
        \hbox to 6.28in { {\it Prepared by: Chaitanya Kharyal\hfill} } %% Fill here
        \vspace{2mm}}
   }
   \end{center}
   \markboth{}{} %% Fill here

\section{A Gentle Start}

\subsection{Posing the problem:}
Suppose with no prior experience about papayas, we arrive at some small Pacific Island, where papayas are a significant ingredient in local diet. Now, we have to learn how to predict whether a papaya is tasty or not based on some of its features. We choose these features to be colour and softness. Now, given a set of papayas, how will we determine whether a papaya is tasty or not.

\subsection{A Formal Model - The Statistical Learning Framework}
We define the following terms:
\begin{itemize}
    \item \textbf{Domain Set:} We can call it the universal set, $\chi$,of the objects that we wish to label. For example in papaya labelling example, it is the set of all papayas.
    \item \textbf{Label set: } It is the set, $Y$, of all possible labels. For example in our papaya labelling case, \{\textit{tasty, non-tasty}\} or if we map it to integers, (\textit{1} for \trextit{tasty} and \textit{0} for \trextit{non-tasty}) \{\textit{0, 1}\}.
    \item \textbf{Training Data: } $S = ((x_1, y_1),...,(x_m,y_m))$ is a finite set in $\chi \times Y$. It is the input that learner has access to. For example, in the papaya example, the previously tasted papayas should make the training set for the learner.
    \item \textbf{The learner's output: } What we want from our learner is a prediction rule, which takes as input a papaya (or features of papaya) and, without tasting it, tries to predict whether the papaya is tasty or non-tasty. This prediction rule is called \textit{hypothesis}. We use $A(S)$ to denote a prediction rule which learning algorithm $A$ returns on training set $S$.
    \item \textbf{The data generation model: } (The learner doesn't know anything about this model) First we assume that the set $\chi$ has some distribution related to it. We denote this prob. dist. over $\chi$ by $\mathcal{D}$. We also assume, for now, that there is some correct labelling function, $f: \chi \to Y$, and $y_i = f(x_i)$ for all $i$. note that the aim of the learner is just to approximate this function $f$ given $S$.
    
\end{itemize}

\end{document}
