% This is a borrowed LaTeX template file for lecture notes for CS267,
% Applications of Parallel Computing, UCBerkeley EECS Department.

% To familiarize yourself with this template, the body contains
% some examples of its use.  Look them over.

\documentclass[a4paper]{article}

%
% ADD PACKAGES here:
%

\usepackage{amsmath,amsfonts,graphicx,multicol}
\usepackage[margin=1in]{geometry}
\usepackage{amsthm}
\usepackage{graphicx}
\graphicspath{ {./images/} }
\usepackage{bm,nicefrac}
\usepackage{hyperref}

\setlength{\parskip}{\baselineskip} % Set space between paras
\setlength{\parindent}{0pt} % No para indentation

\newtheorem{theorem}{Theorem}[section]
\newtheorem{corollary}[theorem]{Corollary}
\newtheorem{lemma}[theorem]{Lemma}
\newtheorem{definition}[theorem]{Definition}
\newtheorem{remark}[theorem]{Remark}
\newtheorem{problem}{Problem}

\begin{document}

\pagestyle{myheadings}
   \thispagestyle{plain}
   \newpage
   \noindent
   \begin{center}
   \framebox
   {
      \vbox{\vspace{2mm}
        \hbox to 6.28in { {\Large \hfill Linear algebra \hfill} } %% Fill here
        \vspace{2mm}
        \hbox to 6.28in { {\it Chaitanya Kharyal, Ishita Vohra\hfill} } %% Fill here
        \vspace{2mm}}
   }
   \end{center}
   \markboth{}{} %% Fill here

\section{Notation:}
We have tried to use uppercase letters ($\mathbf{A}$) for matrices, lowercase bold letters ($\mathbf{x}$) for vectors and lowercase letters ($x$) for scalars (constants and variables). We have tried to stick to this notation as much as possible but there might be some unintentional slips and we apologise for that.

\section{Topics to be covered}
\begin{itemize}
    \item System of linear equations
    \item Row Echelon Form
    \item Reduced Row Echelon Form
    \item Determinant
    \item Inverse and its properties
\end{itemize}

\section{System of linear equations}
\subsection{Representing linear equations in terms of matrices and vectors}
\textit{(covered in T8)}

While solving the system of linear equations using LA, the first question that arises is \textit{"How do we represent Linear Equations using matrices?".} This relatively not so simple question has a simple solution. Consider a system of linear equations in two variables,
\begin{center}
    $ax + by = c$
    
    $dx + ey = f$
\end{center}

In order to represent this simple system of equations, we make a matrix using all the coefficients...
\begin{center}
    $A = \begin{bmatrix}
        a & b \\
        d & e 
    \end{bmatrix}$
\end{center}

We also make two vector using the variables and the constants respectively...
\begin{center}
    $\textbf{x} = \begin{bmatrix}
        x \\
        y 
    \end{bmatrix}$ \hspace{5px} 
    and \hspace{5px}
$\textbf{b} = \begin{bmatrix}
        c \\
        f 
    \end{bmatrix}$ 
\end{center}

Now, our linear equation will become:
\begin{center}
    $A\textbf{x} = \textbf{b}$
\end{center}

\subsection{Column picture of matrix multiplication}
\textit{(covered in T8)}

The fundamental problem in LA is to solve $A\textbf{x} = \textbf{b}$ for $\textbf{x}$. We will now see when is the solution for this equation possible.

We write our linear equation in matrix form,
\begin{center}
    $ 
    \begin{bmatrix}
    a_{1,1} & a_{1,2} & \dots & a_{1,n} \\ 
    a_{2,1} & a_{2,2} & \dots & a_{2,n} \\
    \vdots  & \vdots  &       & \vdots  \\
    a_{m,1} & a_{m,2} & \dots & a_{m,n} 
    \end{bmatrix}
    \begin{bmatrix}
    x_1   \\ 
    x_2   \\
    \vdots\\
    x_n 
    \end{bmatrix} = 
    \begin{bmatrix}
    b_1   \\ 
    b_2   \\
    \vdots\\
    b_m 
    \end{bmatrix}
    
    \implies \begin{bmatrix}
    |   & |       &       & |     \\ 
    c_1 & c_2     & \dots & c_n   \\
    |   & |       &       & |        
    \end{bmatrix}
    \begin{bmatrix}
    x_1   \\ 
    x_2   \\
    \vdots\\
    x_n 
    \end{bmatrix} = 
    \begin{bmatrix}
    b_1   \\ 
    b_2   \\
    \vdots\\
    b_m
    \end{bmatrix}
    $  \dots\dots \textbf{(1)}
\end{center}

Where,
\begin{center}
    $
    c_i = 
    \begin{bmatrix}
    a_{1,i}   \\ 
    a_{2,i}   \\
    \vdots    \\
    a_{m,i} 
    \end{bmatrix}
    $
\end{center}

We can rewrite the equation \textbf{(1)} as,

\begin{center}
    $
    \displaystyle\sum_{i=1}^{n}x_i\textbf{c}_i = \textbf{b} \dots\dots \textbf{(2)}
    $
\end{center}
Verify that (2) is correct.

Here, $\displaystyle\sum_{i=1}^{n}x_i\textbf{c}_i$ is the span of column vectors of $\textbf{A}$, since $\textbf{x}$ is unknown. And for L.H.S and R.H.S of equation (2) to be equal for some $\textbf{x}$, $\textbf{b}$ should belong to the column space of $\textbf{A}$.

If $\textbf{b}$ doesn't belong to the column space of $\textbf{A}$, there will be no solution for the equation. In that case if we want to approximate the solution for this equation, we have to use the trick that was taught in the last tutorial. And the solution, then, comes out to be,
\begin{center}
    $
    \textbf{x} = (\textbf{A}^T\textbf{A})^{-1}\textbf{A}^T\textbf{b}
    $
\end{center}

Also, If $\mathbf{b}$ belongs to the column space of $\textbf{A}$, there might be infinitely many solutions if the matrix $\textbf{A}$ is not full rank.

\begin{definition}
    \textbf{Full Rank: } An $m\times n$ matrix $\mathbf{A}$ is said to be full rank iff $rank(\mathbf{A}) = min(m,n)$.
\end{definition}

Now, let $\mathbf{a}$ be the solution for the linear equation $\mathbf{Ax} = \mathbf{b}$ (obviously $\mathbf{b}$ lies in the column space of $\mathbf{A}$). And, if $\mathbf{a}$ is a solution of the equation, $\mathbf{a} + 0$ will also be a solution. And since we know that we can represent $0$ using columns of $\mathbf{A}$ in infinitely many ways (if $\mathbf{A}$ is not full rank), we know that the equation will have infinitely many solutions.

Eg.

let $\mathbf{a}$ be a solution to the equation to $\mathbf{Ax} = \mathbf{b}$, and $\mathbf{A}$ be a $3 \times 3$ non-full rank (correct word for this would be \textbf{Rank Deficient)} matrix with columns $\mathbf{c_1}, \mathbf{c_2}, \mathbf{c_3}$. Also, Let $\mathbf{c_1} + \mathbf{c_2} + \mathbf{c_3} = 0$.

Therefore,

\begin{center}
$
    \textbf{Aa} = \textbf{b} 
    
    \implies \mathbf{A(a+0)} = \mathbf{b}  
    
    \implies \mathbf{A(a+\epsilon*0)} = \mathbf{b} 
 $   
\end{center}
where $\epsilon$ is a constant,

\begin{center}
$
    \implies \mathbf{A(a+\epsilon*(\mathbf{c_1} + \mathbf{c_2} + \mathbf{c_3}))} = \mathbf{b}
$
\end{center}

Therefore, infinitely many solutions.


\section{Echelon Form}
\textit{(covered in T6)}

Echelon form means matrix is in one of the two states. 
\begin{itemize}
  \item Row Echelon Form
  \item Row Reduced Echelon Form
\end{itemize}


\subsection{What is RE form?}
A matrix is said to be Row Echelon Form if it is:
\begin{itemize}
  \item The first non-zero number from the left (the “leading coefficient” or the "pivot") is always to the right of the first non-zero number in the row above.
  \item Rows consisting of all zeros are at the bottom of the matrix.
\end{itemize}
Examples of matrices in Row Echelon Form:
$ 
    \begin{bmatrix}
    1 & a_{1,2} & a_{1,3} & a_{1,4} \\ 
    0 & 1 &  a_{2,3} & a_{2,4} \\
    0 & 0 & 0  &    1  \\
    0 & 0 & 0 & 0 
    \end{bmatrix}
    ,
    \begin{bmatrix}
    1 & a_{1,2} & a_{1,3} & a_{1,4} & a_{1,5} \\ 
    0 & 1 &  a_{2,3} & a_{2,4} & a_{2,5} \\
    0 & 0 & 1  & a_{3,4} & a_{3,5}  \\
    0 & 0 & 0 & 1 &  a_{4,5} &
    \end{bmatrix}
$

\newpage
\subsection{Reduced RE form}
A matrix is said to be in Reduced Row Echelon Form if it is:
\begin{itemize}
    \item The first non-zero number in the first row (the "leading coefficient" or the "pivot") is the number 1.
    \item The second row also starts with the number 1, which is further to the right than the leading entry in the first row. For every subsequent row, the number 1 must be further to the right.
    \item The leading entry in each row must be the only non-zero number in its column.
    \item Any non-zero rows are placed at the bottom of the matrix.
\end{itemize}
Examples of matrix in RREF are: 
$
\begin{bmatrix}
    1 & 0 & a_{1,3} & 0 \\ 
    0 & 1 &  a_{2,3} & 0 \\
    0 & 0 & 0  &    1  \\
    0 & 0 & 0 & 0 
    \end{bmatrix}
    ,
    \begin{bmatrix}
    1 & 0 & 0 & 0 & a_{1,5} \\ 
    0 & 1 &  0 & 0 & a_{2,5} \\
    0 & 0 & 1  & 0 & a_{3,5}  \\
    0 & 0 & 0 & 1 &  a_{4,5} &
    \end{bmatrix}
$

\section{Solution of equations using RE form}
\textit{(Will be shown with an example in the tutorial)}

If we have matrix in REF or RREF we can solve the system of linear equations in less no of steps and it is computationally less expensive.
We start computing from the last row and proceed further. 

\section{Determinant}
\href{https://brilliant.org/wiki/expansion-of-determinants/}{Determinant Wiki}

\href{https://www.youtube.com/watch?v=Ip3X9LOh2dk}{Determinant Video}

We suggest you to read this Wiki page and watch the Video before proceeding.


\newpage
\section{Inverse (of a matrix) and its properties}
\textit{(finding inverse using EF will be covered in the tutorial)}

In this section, we will discuss some important properties of the matrix inverses and consolidate all the related concepts.

\textbf{Statement1: } A matrix \textbf{A} won't have an Inverse if for some $\mathbf{x}\neq0$, $\mathbf{Ax} = \mathbf{0}$.

\textbf{Explanation: } Suppose, $\mathbf{x}\neq\mathbf{0}$ and, 
\begin{center}
    $\mathbf{Ax} = \mathbf{0}$    
\end{center}
Now, Left multiplying $\mathbf{A^{-1}}$ to both sides

\begin{center}
    $\implies \mathbf{A^{-1}Ax} = \mathbf{0}$\\
    $\implies \mathbf{Ix} = \mathbf{0}$\\
    $\implies \mathbf{x} = \mathbf{0}$
\end{center}
This is a contradiction, therefore, \textbf{Statement1} is true. This can be interpreted in terms of the Null space of $\mathbf{A}$. Since $\mathbf{Ax} = \mathbf{0}$ is true only for $\mathbf{x} = \mathbf{0}$, $Null(\mathbf{A})$  will contain only $\mathbf{0}$ vector.

This result can also be interpreted using column picture of Matrix Multiplication,

We can rewrite equation \textbf{(2)} for $\mathbf{b} = \mathbf{0},$
\begin{center}
    $
    \displaystyle\sum_{i=1}^{n}x_i\textbf{c}_i = \textbf{0} \dots\dots
    $
    \textbf{(3)}
\end{center}

This brings us back to the definition of \textbf{Linear Independence}, and since $\mathbf{A}$ has an inverse if $\mathbf{x} = \mathbf{0}$. This implies if $\mathbf{A^{-1}}$ exists, equation \textbf{(3)} can be $\mathbf{0}$ iff all the $x_i$s are $0$, which in turn implies that all $c_i$s are Linearly Independent. And, in contrast, if equation \textbf{(3)} holds true for some non-zero $x_i$s, columns of $\mathbf{A}$ are Linearly Dependent, $\mathbf{A^{-1}}$ doesn't exist and the $Det(\mathbf{A}) = 0$.

You can, now, see that how all these concepts are parallel and yet so much inter-related. 

\textbf{Statement2: }If the columns of $\mathbf{A}$ are Independent, then $\mathbf{A^{T}A}$ is Invertible.

\textbf{Explanation: }Suppose,
\begin{center}
    $\mathbf{A^TAx} = \mathbf{0}$
\end{center}
Dot product with $\mathbf{x^T}$ on both sides,
\begin{center}
    $
    \implies \mathbf{x^TA^TAx} = 0 \dots\dots \textrm{(note that 0 is not in bold here)}

    \implies \mathbf{(Ax)^TAx} = 0
    $
\end{center}

This says that $\mathbf{Ax}$ is perpendicular to itself, which is possible only when $\mathbf{Ax} = \mathbf{0}$. Now, Since columns of $\mathbf{A}$ are Independent, following from the above discussion, $\mathbf{x} = \mathbf{0}$. Therefore, using statement \textbf{(1)}, it follows that $\mathbf{A^TA}$ is invertible.

\newpage
\section{Exploring the subject further}
We \textbf{strongly recommend} you to look into the \textbf{formulas and basic properties} for determinant and inverse.

Related links:
\begin{enumerate}
    \item https://www.youtube.com/watch?v=srxexLishgY
    \item https://www.youtube.com/watch?v=23LLB9mNJvc
    \item https://www.youtube.com/watch?v=uQhTuRlWMxw
    \item https://brilliant.org/wiki/matrix-inverse/
\end{enumerate}

\section{Problems}
\begin{problem}
    
    \textbf{[Graded Assignment, to be submitted before next tutorial]} Give the value of $\lambda$ for which the following system of linear equations have infinitely many solutions.
    \begin{center}
        $
        2x + 2y + 7z = 7\\
        3x + (\lambda+9)y + (\lambda+11)z = 10 \\
        3x + (\lambda+4)y + 11z = 10
        $
    \end{center}
\end{problem}

\end{document}
